\setchapterimage[6cm]{seaside}
\setchapterpreamble[u]{\margintoc}
\chapter{Page Design}
\labch{layout}

\section{Headings}

So far, in this document I used two different styles for the chapter 
headings: one has the chapter name, a rule and, in the margin, the 
chapter number; the other has an image at the top of the page, and the 
chapter title is printed in a box (like for this chapter). There is one 
additional style, which I used only in the appendix 
(\vrefpage{appendix}); there, the chapter title is enclosed in two 
horizontal rules, and the chapter number (or letter, in the case of the 
appendix) is above it.\sidenote{To be honest, I do not think that mixing 
heading styles like this is a wise choice, but in this document I did it 
only to show you how they look.}

Every book is unique, so it makes sense to have different styles from 
which to choose. Actually, it would be awesome if whenever a 
\Class{kao}-user designs a new heading style, he or she added it to the 
three styles already present, so that it will be available for new users 
and new books.

The choice of the style is made simple by the \Command{setchapterstyle} 
command. It accepts one option, the name of the style, which can be: 
\enquote{plain}, \enquote{kao}, or \enquote{lines}.\sidenote{Plain is 
the default \LaTeX\xspace title style; the other ones are self 
explanatory.} If instead you want the image style, you have to use the 
command \Command{setchapterimage}, which accepts the path to the image 
as argument; you can also provide an optional parameter in square 
brackets to specify the height of the image.

Let us make some examples. In this book, I begin a normal chapter with 
the lines:

\begin{lstlisting}
\setchapterstyle{kao}
\setchapterpreamble[u]{\margintoc}
\chapter{Title of the Chapter}
\labch{title}
\end{lstlisting}

In Line 1 I choose the style for the title to be \enquote{kao}. Then, I 
specify that I want the margin toc. The rest is ordinary administration 
in \LaTeX, except that I use my own \Command{labch} to label the 
chapter. Actually, the \Command{setchapterpreamble} is a standard 
\KOMAScript\xspace one, so I invide you to read about it in the KOMA 
documentation. Once the chapter style is set, it holds until you change 
it.\sidenote{The \Command{margintoc} has to be specified at every 
chapter. Perhaps in the future this may change; it all depends on how 
this feature will be welcomed by the users, so keep in touch with me if 
you have preferences!} Whenever I want to start a chapter with an image, 
I simply write:

\begin{lstlisting}
\setchapterimage[7cm]{path/to/image.png} % Optionally specify the height
\setchapterpreamble[u]{\margintoc}
\chapter{Catchy Title} % No need to set a chapter style
\labch{catchy}
\end{lstlisting}

If you prefer, you can also specify the style at the beginning of the 
main document, and that style will hold until you change it again.

\section{Headers \& Footers}

Headers and footers in \KOMAScript\xspace are handled by the 
\Package{scrlayer-scrpage} package. There are two basic style: 
\enquote{scrheadings} and \enquote{plain.scrheadings}. The former is 
used for normal pages, whereas the latter is used in title pages (those 
where a new chapter starts, for instance) and, at least in this book, in 
the front matter. At any rate, the style can be changed with the 
\Command{pagestyle} command, \eg 
\lstinline|\pagestyle{plain.scrheadings}|.

In both stles, the footer is completely empty. In plain.scrheadings, 
also the header is absent (otherwise it wouldn't be so plain\ldots), but 
in the normal style the design is reminescent of the \enquote{kao} style 
for chapter titles.

\begin{kaobox}[frametitle=To Do]
The \Option{twoside} class option is still unstable and may lead to 
unexpected behaviours. As always, any help will be greatly appreciated.
\end{kaobox}

\section{Table of Contents}

Another important part of a book is the table of contents. By default, 
in \Class{kaobook} there is an entry for everything: list of figures, 
list of tables, bibliographies, and even the table of contents itself. 
Not everybody might like this, so we will provide a description of the 
changes you need to do in order to enable or disable each of these 
entries. In the following \reftab{tocentries}, each item corresponds to 
a possible entry in the \acrshort{tocLabel}, and its description is the 
command you need to provide to have such entry. These commands are 
specified in the attached \href{style/style.sty}{style 
package},\sidenote{In the same file, you can also choose the titles of 
these entries.} so if you don't want the entries, just comment the 
corresponding lines.

Of course, some packages, like those for glossaries and indices, will 
try to add their own entries.\marginnote{In a later section, we will see 
how you can define your own floating environment, and endow it with an 
entry in the \acrshort{tocLabel}.} In such cases, you have to follow the 
instructions specific to that package. Here, since we have talked about 
glossaries and notations in \refch{references}, we will biefly see how 
to configure them.

\begin{table}
\footnotesize
\caption{Commands to add a particular entry to the table of contents.}
\labtab{tocentries}
\begin{tabular}{ l l }
	\toprule
	Entry & Command to Activate \\
	\midrule
	Table of Contents & \lstinline|\setuptoc{toc}{totoc}| \\
	List of Figs and Tabs & \lstinline|\PassOptionsToClass{toc=listof}{\@baseclass}| \\
	Bibliography & \lstinline|\PassOptionsToClass{toc=bibliography}{\@baseclass}| \\
	\bottomrule
\end{tabular}
\end{table}

For the \Package{glossaries} package, use the \enquote{toc} option when 
you load it: \lstinline|\usepackage[toc]{glossaries}|. For 
\Package{nomencl}, pass the \enquote{intoc} option at the moment of 
loading the package. Both \Package{glossaries} and \Package{nomencl} are 
loaded in the attached \href{style/packages.sty}{\enquote{packages} 
package}.

Additional configuration of the table of contents can be performed 
through the packages \Package{etoc}, which is loaded because it is 
needed for the margintocs, or the more traditional \Package{tocbase}. 
Read the respective documentations if you want to be able to change the 
default \acrshort{tocLabel} style.\sidenote[][*-1]{(And please, send me 
a copy of what you have done, I'm so curious!)}

\section{Page Layout}

Besides the page style, you can also change the width of the content of 
a page. This is particularly useful for pages dedicated to part titles, 
where having the 1.5-column layout might be a little awkward, or for 
pages where you only put figures, where it is important to exploit all 
the available space.

In practice, there are two layouts: \enquote{wide} and \enquote{margin}. 
The former suppresses the margins and allocates the full page for 
contents, while the latter is the layout used in most of the pages of 
this book, including this one. The wide layout is also used 
automatically in the front and back matters.

To change page layout, use the \Command{pagelayout} command. For 
example, when I start a new part, I write:

\begin{lstlisting}
\pagelayout{wide}
\addpart{Title of the New Part}
\pagelayout{margin}
\end{lstlisting}

\section{Numbers \& Counters}

In this short section we shall see how dispositions, sidenotes and 
figures are numbered in the \Class{kaobook} class.

By default, dispositions are numbered up to the section. This is 
achieved by setting: \lstinline|\setcounter{secnumdepth}{1}|.

The sidenotes counter is the same across all the document, but if you 
want it to reset at each chapter, just uncomment the line

\begin{lstlisting}[style=kaolstplain]
\counterwithin*{sidenote}{chapter}
\end{lstlisting}

in the \Package{styles/style.sty} package provided by this class.

Figure and Table numbering is also per-chapter; to change that, use 
something like:

\begin{lstlisting}[style=kaolstplain]
\renewcommand{\thefigure}{\arabic{section}.\arabic{figure}}
\end{lstlisting}

\section{White Space}

One of the things that I find most hard in \LaTeX\xspace is to finely 
tune the white space around objects. There are not fixed rules, each 
object needs its own adjustment. Here we shall see how some spaces are 
defined at the moment in this class.\marginnote{Attention! This section 
may be incomplete.}

\textbf{Space around figures and tables}

\begin{lstlisting}[style=kaolstplain]
\renewcommand\FBaskip{.4\topskip}
\renewcommand\FBbskip{\FBaskip}
\end{lstlisting}

\textbf{Space around captions}

\begin{lstlisting}[style=kaolstplain]
\captionsetup{
	aboveskip=6pt,
	belowskip=6pt
}
\end{lstlisting}

\textbf{Space around displays (\eg equations)}

\begin{lstlisting}[style=kaolstplain]
\setlength\abovedisplayskip{6pt plus 2pt minus 4pt}
\setlength\belowdisplayskip{6pt plus 2pt minus 4pt}
\abovedisplayskip 10\p@ \@plus2\p@ \@minus5\p@
\abovedisplayshortskip \z@ \@plus3\p@
\belowdisplayskip \abovedisplayskip
\belowdisplayshortskip 6\p@ \@plus3\p@ \@minus3\p@
\end{lstlisting}

\documentclass{article}

\usepackage{fontspec}

\setmainfont{FreeSerif}
% \setmainfont{Linux Libertine}
\setsansfont{FreeSans}
\setmonofont{FreeMono}

\usepackage{polyglossia}
\setdefaultlanguage{english}
\setotherlanguages{greek}

\sloppy

\begin{document}

Hello there. Nice to meet you...
Update wanted yet. Using Linux Libertine 0.
Updated after sans and mono.  And then added polyglossia, \textsf{english}, (Ελληνικά ...), greek. \textbf{\textit{accident}}, and \textbf{\textit{ατύχημα.}}

\begin{greek}
  Ελληνικά.

  Καλημέρα Μάνο,

    ας βρεθούμε μεθαύριο \textsf{Παρασκευή} στις 14:00, γιατί αύριο πρέπει να βοηθήσω σε δουλειές τη μάνα μου να μη βγαίνει με τη ζέστη.

    Μπορούμε γενικότερα να αντικαταστήσουμε τις συναντήσεις που προτείνω για Δευτέρα, με αντίστοιχες τις Παρασκευές κάθε εβδομάδα.


ΟΚ για αύριο Παρασκευή 14:00.
Επίσης και για Δευτέρες-παρασκευές από δώ κα ιπέρα

    Όσο \textit{περισσότερο} ξανα-επισκέπτομαι το Pure \texttt{Data} αυτές τις μέρες, τόσο \texttt{συνειδητοποιώ} τη δυσκολία του εγχειρήματος μας.

    Σίγουρα μια safe θεώρηση είναι ένα step-by-step tutorial για το κάθε \textbf{προγραμματιστικό} περιβάλλον.

\textit{Σίγουρα μια safe θεώρηση είναι ένα step-by-step tutorial για το κάθε προγραμματιστικό περιβάλλον.}


Ας το δούμε μαζί. Μπορούμε να κάνουμε μαι γενική σύγκριση σε θεωρητικό επίπεδο σε ένα ξεχωριστό κεφαλαιο,
και μετά να κάνουμε μια λίστα από βασικές εφαρμογες πχ. additive synthesis που μπορούν να γίνουν και με τια
δυο περιβάλλοντα, και να δείξουμε πως γίνεται η καθε εφαρμογή στο καθένα, σε αντοπαράθεση και σύγκριση.
Αρκει να φτιάξουμε την λίστα και να αρχίσουμε, και θα δούμε.

    Πιο δύσκολο, εντούτοις ενδιαφέρον, να είναι κάθε Κεφάλαιο project based, όπου οι αναγνώστες να φτάνουν στο ίδιο αποτέλεσμα από δύο δρόμους.


Νομ'ιζω αυτό εννοοώ και γώ παραπάνω. Ας αρχίσουμε και βλέπουμε.
\end{greek}

\end{document}

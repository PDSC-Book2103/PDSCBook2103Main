
	You can edit this page to suit your needs. For instance, here we have a no copyright statement, a colophon and some other information. This page is based on the corresponding pag
I am slowly beginning to get nervous whether the thinky is compiled.
This building from scratch is slow but it will succeed.
Made glossaries. Made nomenclature.
\(Ελληνικά ...\), greek. \textbf{\textit{accident}}, and \textgreek{\textbf{\textit{ατύχημα.}}}

Now trying to fix this by enclosing it in a textgreek statement:
The following word is going to be in Greek: \textgreek{Ελληνικά}. And continuing in English...

Some more greek here: \textgreek{ελληνικά.  Μπράβο.}

\begin{greek}
  Ελληνικά.

  Καλημέρα Μάνο,

    ας βρεθούμε μεθαύριο \textsf{Παρασκευή} στις 14:00, γιατί αύριο πρέπει να βοηθήσω σε δουλειές τη μάνα μου να μη βγαίνει με τη ζέστη.

    Μπορούμε γενικότερα να αντικαταστήσουμε τις συναντήσεις που προτείνω για Δευτέρα, με αντίστοιχες τις Παρασκευές κάθε εβδομάδα.

    ok let's try some English here for fun.

ΟΚ για αύριο Παρασκευή 14:00.
Επίσης και για Δευτέρες-παρασκευές από δώ κα ιπέρα

    Όσο \textit{περισσότερο} ξανα-επισκέπτομαι το Pure \texttt{Data} αυτές τις μέρες, τόσο \texttt{συνειδητοποιώ} τη δυσκολία του εγχειρήματος μας.

    Σίγουρα μια safe θεώρηση είναι ένα step-by-step tutorial για το κάθε \textbf{προγραμματιστικό} περιβάλλον.

\textit{Σίγουρα μια safe θεώρηση είναι ένα step-by-step tutorial για το κάθε προγραμματιστικό περιβάλλον.}


Ας το δούμε μαζί. Μπορούμε να κάνουμε μαι γενική σύγκριση σε θεωρητικό επίπεδο σε ένα ξεχωριστό κεφαλαιο,
και μετά να κάνουμε μια λίστα από βασικές εφαρμογες πχ. additive synthesis που μπορούν να γίνουν και με τια
δυο περιβάλλοντα, και να δείξουμε πως γίνεται η καθε εφαρμογή στο καθένα, σε αντοπαράθεση και σύγκριση.
Αρκει να φτιάξουμε την λίστα και να αρχίσουμε, και θα δούμε.

    Πιο δύσκολο, εντούτοις ενδιαφέρον, να είναι κάθε Κεφάλαιο project based, όπου οι αναγνώστες να φτάνουν στο ίδιο αποτέλεσμα από δύο δρόμους.


Νομ'ιζω αυτό εννοοώ και γώ παραπάνω. Ας αρχίσουμε και βλέπουμε.
\end{greek}

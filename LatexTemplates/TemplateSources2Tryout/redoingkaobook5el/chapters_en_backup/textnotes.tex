\setchapterpreamble[u]{\margintoc}
\chapter{Margin Stuff}

Sidenotes are a distinctive feature of all 1.5-column-layout books. 
Indeed, having wide margins means that some material can be displayed 
there. We use margins for all kind of stuff: sidenotes, marginnotes, 
small tables of contents, citations, and, why not?, special boxes and 
environments.

\section{Sidenotes}

Sidenotes are like footnotes, except that they go in the margin, where 
they are more readable. To insert a sidenote, just use the command 
\Command{sidenote\{Text of the note\}}. You can specify a 
mark\sidenote[O]{This sidenote has a special mark, a big O!} with \\ 
\Command{sidenote[mark]\{Text\}}, but you can also specify an offset, 
which moves the sidenote upwards or downwards, so that the full syntax is:

\begin{lstlisting}[style=kaolstplain]
\sidenote[mark][offset]{Text}
\end{lstlisting}

If you use an offset, you always have to add the brackets for the mark, 
but they can be empty.\sidenote{If you want to know more about the usage 
of the \Command{sidenote} command, read the documentation of the 
\Package{sidenotes} package.}

In \Class{kaobook} we copied a feature from the \Package{snotez} 
package: the possibility to specify a multiple of \Command{baselineskip} 
as an offset. For example, if you want to enter a sidenote with the 
normal mark and move it upwards one line, type:

\begin{lstlisting}[style=kaolstplain]
\sidenote[][*-1]{Text of the sidenote.}
\end{lstlisting}

As we said, sidenotes are handled through the \Package{sidenotes} 
package, which in turn relies on the \Package{marginnote} package.

\section{Marginnotes}

This command is very similar to the previous one. You can create a 
marginnote with \Command{marginnote[offset]\{Text\}}, where the offset 
argument can be left out, or it can be a multiple of 
\Command{baselineskip},\marginnote[-1cm]{While the command for margin 
notes comes from the \Package{marginnote} package, it has been redefined 
in order to change the position of the optional offset argument, which 
now precedes the text of the note, whereas in the original version it 
was at the end. We have also added the possibility to use a multiple of 
\Command{baselineskip} as offset. These things were made only to make 
everything more consistent, so that you have to remember less things!} 
\eg

\begin{lstlisting}[style=kaolstplain]
\marginnote[-12pt]{Text} or \marginnote[*-3]{Text}
\end{lstlisting}

\begin{kaobox}[frametitle=To Do]
A small thing that needs to be done is to renew the \Command{sidenote} 
command so that it takes only one optional argument, the offset. The 
special mark argument can go somewhere else. In other words, we want the 
syntax of \Command{sidenote} to resemble that of \Command{marginnote}.
\end{kaobox}

We load the packages \Package{marginnote}, \Package{marginfix} and 
\Package{placeins}. Since \Package{sidenotes} uses \Package{marginnote}, 
what we said for marginnotes is also valid for sidenotes. Side- and 
margin- notes are shifted slightly upwards 
(\Command{renewcommand\{\textbackslash marginnotevadjust\}\{3pt\}}) in 
order to allineate them to the bottom of the line of text where the note 
is issued.

\section{Footnotes}

Even though they are not displayed in the margin, we will discuss about 
footnotes here, since sidenotes are mainly intended to be a replacement 
of them. Footnotes force the reader to constantly move from one area of 
the page to the other. Arguably, marginnotes solve this issue, so you 
should not use footnotes. Nevertheless, for completeness, we have left 
the standard command \Command{footnote}, just in case you want to put a 
footnote once in a while.\footnote{And this is how they look like. 
Notice that in the PDF file there is a back reference to the text; 
pretty cool, uh?}

\section{Margintoc}

Since we are talking about margins, we introduce here the 
\Command{margintoc} command, which allows one to put small table of 
contents in the margin. Like other commands we have discussed, 
\Command{margintoc} accepts a parameter for the vertical offset, like 
so: \Command{margintoc[offset]}.

The command can be used in any point of the document, but we think it 
makes sense to use it just at the beginning of chapters or parts. In 
this document I make use of a \KOMAScript\xspace feature and put it in 
the chapter preamble, with the following code:

\marginnote{The font used in the margintoc is the same as the one for 
	the chapter entries in the main table of contents at the beginning 
	of the document.}

\begin{lstlisting}[style=kaolstplain]
\setchapterpreamble[u]{\margintoc}
\chapter{Chapter title}
\end{lstlisting}

\section{Marginlisting}

On some occasions it may happen that you have a very short piece of code 
that doesn't look good in the body of the text because it breaks the 
flow of narration: for that occasions, you can use a 
\Environment{marginlisting}. The support for this feature is still 
limited, especially for the captions, but you can try the following 
code:

\begin{marginlisting}[-1.35cm]
	\caption{An example of a margin listing.}
	\vspace{0.6cm}
	\begin{lstlisting}[language=Python,style=kaolstplain]
print("Hello World!")
	\end{lstlisting}
\end{marginlisting}

\begin{verbatim}
\begin{marginlisting}[-0.5cm]
	\caption{My caption}
	\vspace{0.2cm}
	\begin{lstlisting}[language=Python,style=kaolstplain]
	... code ...
	\end{lstlisting}
\end{marginlisting}
\end{verbatim}

Unfortunately, the space between the caption and the listing must be 
adjusted manually; if you find a better way, please let me know.

Not only textual stuff can be displayed in the margin, but also figures. 
Those will be the focus of the next chapter.
